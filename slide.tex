\documentclass[dvipdfmx,11pt]{beamer}
\setbeamertemplate{frametitle continuation}[from second][]
\usepackage{bxdpx-beamer}
\usepackage{pxjahyper}
\usepackage{minijs}
\usefonttheme{professionalfonts}
\usepackage{amsmath,amssymb,nccmath}
\usepackage{bm}
\usepackage{graphicx}
\usepackage{comment}
\usepackage{here}
\usepackage{ascmac}
\usepackage{mathrsfs}
\usepackage[absolute,overlay]{textpos}
\usepackage {capt-of} 
\usepackage{color}
\setbeamertemplate{caption}[numbered]
\usepackage{braket}
\usepackage{caption}
\captionsetup[figure]{font=scriptsize,labelfont=scriptsize}
\setbeamertemplate{footline}[frame number]
\usepackage{appendixnumberbeamer}
\usepackage{color}
\definecolor{myblue}{rgb}{0.2,0.2,0.7}

\renewcommand{\figurename}{図}
\renewcommand{\kanjifamilydefault}{\gtdefault}
\usetheme[    
    block=fill, % ブロックに背景をつける
    progressbar=foot, % 各スライドの下にプログレスバー
    numbering=fraction % 合計ページ数を表示
]{metropolis}

\title{クリーンな2次元超伝導体における量子渦にかかる力の微視的理論}
\subtitle{Microscopic theory of force on quantum vortex in two-dimensional clean superconductors}
\author{牧野 舜}
\institute{Hoge University}
\date{}

\begin{document}
\begin{frame}\frametitle{}
    \titlepage
\end{frame}

\AtBeginSection[]
{
    \begin{frame}
        \frametitle{Outline}
        \tableofcontents[currentsection]
    \end{frame}
}

\section{超伝導と量子渦}
\begin{frame}
    \frametitle{背景知識:超伝導と量子渦}

    \begin{itemize}
        \item
              第I種超伝導体:完全反磁性(マイスナー状態)
        \item
              第II種超伝導体:混合状態では,量子渦(vortex)が現れ,量子化された磁束の侵入を許す.
    \end{itemize}
\end{frame}

\section{磁束の量子化}
\begin{frame}
    \frametitle{背景知識:超伝導と量子渦}

    \begin{itemize}
        \item
              波動関数 $\Psi(r,\theta)=\Psi_{0}(r)e^{-i\theta}$の一価性から磁束は量子化される.

              \begin{equation}
                  \int_{S} \bm{B}\cdot\mathrm{d}\bm{S} = \frac{h}{2e}n \ \ (n=0,\pm1,\dots) \notag
              \end{equation}
    \end{itemize}
\end{frame}

\section{まとめ}
\begin{frame}
    \frametitle{まとめ}

    \begin{itemize}
        \item まとめ
    \end{itemize}
\end{frame}

\end{document}

